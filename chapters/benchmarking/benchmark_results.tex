\section{Benchmark results}
\label{sec:benchmark_results}
This section will present the benchmark results for intent classification using micro f1 score.
An explanation for the differences in averages for the f1 score is presented in section~\ref{subsec:methodology}.
Here micro f1 scores are used to allow comparing results with~\citet{braun2017}.
The results are listed in table~\ref{tab:benchmark_comparison}.

\begin{table}[!htbp]
    \centering
    \begin{tabular}{l c c c c}
        \textbf{System} & \textbf{Source} & \textbf{AskUbuntu} & \textbf{Chatbot} & \textbf{WebApplications} \\
        \hline
        Rasa:0.5-mitie & see section~\ref{sec:recalculations_rasa} & 0.862 & 0.981 & 0.746 \\
        Microsoft LUIS & see section~\ref{sec:recalculations_luis} & 0.899 & 0.981 & 0.814 \\
        Watson Conversation (2017) & see section~\ref{sec:recalculations_watson} & 0.917 & 0.972 & 0.831 \\
        Rasa:0.13.7-mitie & \textsc{bench} & 0.881 & & 0.763 \\
        Rasa:0.13.8-spacy & \textsc{bench} & 0.853 & 0.981 & 0.627 \\
        Watson Conversation (2018) & \textsc{bench} & 0.881 & 0.934 & 0.831 \\
        \hline
    \end{tabular}
    \caption{Micro f1 scores for intent classification.
    One score is missing due to a bug in \textsc{bench}.}
    \label{tab:benchmark_comparison}
\end{table}

The paper remarks that ``For our two corpora, LUIS showed the best results, however, the open source alternative RASA could achieve similar results.''
When considering only intents this does not hold.
Watson Conversation has very similar results, and in fact slightly higher scores on two out of three datasets.
The MITIE back-end outperforms the spaCy back-end in terms of accuracy.
This would not support the choice of Rasa to depreciate MITIE.
It is expected to be caused by the facts that training MITIE takes more time than spaCy and MITIE tends to freeze during training.
Interesting to see is that the accuracy for Watson Conversation has dropped.
The cause can only be guessed since IBM does not provide information about the Watson back-end.
It could be that the calculations for \textsc{bench} and the paper differ.
Alternatively it could be that the back-end for Watson has changed.
The datasets under consideration are small, so it might be that Watson has chosen a back-end better suited for large datasets.
Note that IBM is aimed at large companies.
These companies have the resources for creating lots of training examples.