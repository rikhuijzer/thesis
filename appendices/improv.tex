\chapter{\textsc{improv}}
\label{ch:improv}
% intro
The project aimed to improve accuracy is called \textsc{improv} and available on Github (\url{https://github.com/rikhuijzer/improv}).
One warning for people interested in reading or using this code is that it is in need of refactoring.
The code is cloned from the Google BERT code, built on the Google TensorFlow library, as provided by the researchers~\citep{devlin2018github}.

% pytorch
Alternatively, code is available for the PyTorch library~\citep{wolf2018github}.
The PyTorch implementation is under active development unlike the TensorFlow implementation and includes more functionality.
Features include Multi-GPU, distributed and 16-bits training.
These allow the model to be trained more easily on GPUs by reducing and distributing the memory used by the model.
BERT contains various models including \bert{base} and \bert{large}.
Since Google Colab does not provide a multi-GPU set-up, we need to use a TPU.
This is not yet supported by PyTorch~\citep{wolf2018github}.

\section{Usage}
\label{sec:improv_usage}
The \textsc{improv} code can partially be executed on a local system.
However, training the model requires at least one enterprise grade GPU.
This is discussed in Section~\ref{subsec:training}.
GPUs and TPUs are provided for free by Google Colab~\citep{google2019colab}.
Using this code means importing one of IPython Notebooks from the \textsc{improv} repository in Colab.
Hyperparameters can be set in the Notebook after which the code can be executed.
The Notebook require a Google Account combined with a paid Google Cloud Bucket.
The Bucket is used to store the trained checkpoints created by training the model.
Newer runs listed in the `runs' folder in the Github repository depend on \textsc{improv}, \textsc{nlu\_datasets} and \textsc{rasa\_nlu}.
The dependencies list the used version in the Notebook.
When errors occur make sure that the correct versions are cloned or installed.

\iffalse
\section{Overview}
\label{sec:improv_overview}
Since the code is in need of refactoring this section will only describe the most important parts of the project.

\fi

% todo: describe estimator API and why it does not work