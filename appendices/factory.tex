\chapter{Lazy evaluation demonstration}
\label{ch:demonstration}

This appendix demonstrates the effect of using iterators instead of regular collections.
The code demonstrates this by processing some fictional raw materials to a chair.
The first is function called \textsc{ford} is similar to a Ford factory around 1915.
Here each part of the assembly line just keeps producing items as long as there is input coming in.
After a while the other parts of the assembly line start processing the items and discover a fault in the items.
One problem of this way of working is that the factory now has a pile of incorrect items in their stock.

The second function called \textsc{toyota} is similar to a Toyota factory after 1960.
Here just-in-time (JIT) manufacturing is used as developed by Toyota~\citep{ohno1988toyota}.
Each item is processed only when the next step in the process makes a request for this item.

\section{Benefits}
\label{sec:benefits}
Using JIT makes sense in computer programs for the following reasons.
It saves memory.
In each step in the process we only store one intermediate result instead of all intermediate results.

It can detect bugs earlier.
Suppose you got a combination of processing steps, lets call them \textsc{f} and \textsc{g} and you apply them to 100 items.
In \textsc{f} we send some object to a system and get a response.
In \textsc{h} we store the response of this API call.
Suppose there is a bug in \textsc{h}, lets say the file name is incorrect.
Suppose this is not covered in the tests and we decide to run our program to get all the results we want.
Using an approach similar to \textsc{ford} the program crashes after doing 100 executions of \textsc{f} and \textsc{g}.
This means that the program executed 100 API calls.
Using \textsc{toyota} the program crashes after just one API call.
Here \textsc{ford} has in essence wasted 99 API calls.

It does not make assumptions for the caller.
Suppose some function \textsc{k} returns an iterable and is called by \textsc{l}.
The function \textsc{l} can now decide how it wants to use the iterable.
For example it can be casted to unique values via \textsc{set} or it can partly be evaluated by using \textsc{any}.

\section{Code}
\label{sec:code}
\begin{verbatim}
from typing import List, NamedTuple
""" See README.md """

Wood = NamedTuple('Wood', [('id', int)])
Chair = NamedTuple('Chair', [('id', int)])


materials = [Wood(0), Wood(1), Wood(2)]


def ford():
    """ Processing all the items at once and going to the next step. """
    def remove_faulty(items: List[Wood]) -> List[Wood]:
        out = []
        for material in items:
            print('inspecting {}'.format(material))
            if material.id != 1:
                out.append(material)
        return out

    def process(items: List[Wood]) -> List[Chair]:
        out = []
        for material in items:
            print('processing {}'.format(material))
            out.append(Chair(material.id))
        return out

    filtered = remove_faulty(materials)
    processed = process(filtered)
    print('Result of ford(): {}'.format(processed))


def toyota():
    """ Processing all the items one by one. """
    def is_not_faulty(material: Wood) -> bool:
        print('inspecting {}'.format(material))
        return material.id != 1

    def process(material: Wood) -> Chair:
        print('processing {}'.format(material))
        return Chair(material.id)

    filtered = filter(is_not_faulty, materials)
    processed = list(map(process, filtered))
    print('Result of toyota(): {}'.format(processed))


if __name__ == '__main__':
    ford()
    print()
    toyota()
\end{verbatim}

\section{Output}
\label{sec:output}
The output for the program is as follows.

\begin{verbatim}
inspecting Wood(id=0)
inspecting Wood(id=1)
inspecting Wood(id=2)
processing Wood(id=0)
processing Wood(id=2)
Result of ford(): [Chair(id=0), Chair(id=2)]

inspecting Wood(id=0)
processing Wood(id=0)
inspecting Wood(id=1)
inspecting Wood(id=2)
processing Wood(id=2)
Result of toyota(): [Chair(id=0), Chair(id=2)]
\end{verbatim}

This demonstrates that iterator elements are only executed when called.
